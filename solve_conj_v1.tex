\documentclass[11pt,letterpaper,reqno]{amsart}
\usepackage{tikz}
\usetikzlibrary{positioning, shapes.geometric, arrows.meta, calc, positioning}
\usepackage{amssymb}
\usepackage{amsmath}
\usepackage{amsthm}
\usepackage{amsfonts}
\usepackage{bbm}
\usepackage{enumitem} 
\usepackage{pgfplots}
\pgfplotsset{compat=1.18} 
\usepackage{booktabs}
\usepackage{graphicx}
\usepackage[T1]{fontenc}
\usepackage{doi}
\usepackage{float} 
\usepackage{comment} 
\addtolength{\hoffset}{-1.5cm}\addtolength{\textwidth}{3cm}
\addtolength{\voffset}{-1cm}\addtolength{\textheight}{2cm}

\usepackage{bookmark}
\usepackage{hyperref}
\hypersetup{pdfstartview={FitH}}
\newcommand{\C}{\mathbb{C}}
\newcommand{\cE}{\mathcal{E}}
\newcommand{\norm}[1]{\lVert #1 \rVert}
\newcommand{\abs}[1]{| #1 |}
\newcommand{\bv}{\mathbf{v}}
\newcommand{\bw}{\mathbf{w}}
\newcommand{\tr}{\operatorname{Tr}}
\DeclareMathOperator{\rank}{rank}




\newtheorem{theorem}{Theorem}[section]
\newtheorem{lemma}[theorem]{Lemma}
\newtheorem{proposition}[theorem]{Proposition}
\newtheorem{corollary}[theorem]{Corollary}
\newtheorem{claim}{Claim}
\newtheorem{question}[theorem]{Question}
\newtheorem{problem}[theorem]{Problem}
\newtheorem{conjecture}[theorem]{Conjecture}
\theoremstyle{definition}
\newtheorem{example}[theorem]{Example}
\newtheorem{remark}[theorem]{Remark}
\newtheorem{definition}[theorem]{Definition}
\numberwithin{equation}{section}
\newcommand{\NN}{\mathbb{N}}
\newcommand{\taufunc}{\tau}
\newcommand{\omegap}{\omega}
\newcommand{\ord}{\operatorname{ord}}
\newcommand{\R}{\mathbb{R}}        % real numbers
\newcommand{\E}{\mathbb{E}}        % expectation
\newcommand{\Var}{\mathrm{Var}}    % variance
\newcommand{\Cov}{\operatorname{Cov}}
\newcommand{\PP}{\mathbb{P}}     % probability
\newcommand{\eps}{\varepsilon}     % epsilon
\newcommand{\ind}{\mathbf{1}}      % indicator function
\newcommand{\seq}[1]{\left(#1\right)} % sequence
\newcommand{\lcm}{\operatorname{lcm}}
\newcommand{\seqnum}[1]{\href{https://oeis.org/#1}{\rm \underline{#1}}}
\makeatother


\begin{document}

\title[The self-generating consecutive-sum sequence]{A note on the self-generating consecutive-sum greedy sequence}


\author[Quanyu Tang]{Quanyu Tang}
\address{School of Mathematics and Statistics, Xi'an Jiaotong University, Xi'an 710049, P. R. China}
\email{tang\_quanyu@163.com}

\date{\today}
% \subjclass[2020]{xxx}

% \keywords{xxx}

\begin{abstract}
Let $(a_n)_{n \geq 1}$ be the greedy self-generating sequence with $a_1=1$, $a_2=2$, and for
$k\ge3$ let $a_k$ be the least integer $>a_{k-1}$ that can be written as a sum of
at least two consecutive earlier terms.  Writing $b_n:=a_n-n$, we prove that
$(b_n)_{n \geq 1}$ is nondecreasing and unbounded. In particular, the sequence omits
infinitely many positive integers, settling a conjecture recorded in the OEIS entry A005243.
\end{abstract}



\maketitle


\section{Introduction}

A \emph{self-generating} process introduced by Hofstadter starts from $1,2$ and
repeatedly adjoins all sums of at least two consecutive previous terms.
Equivalently, one may define a strictly increasing sequence $(a_n)_{n\ge1}$ by
$a_1=1$, $a_2=2$, and for each $k\ge3$ letting $a_k$ be the least integer
$a_k>a_{k-1}$ admitting a representation
\[
a_k=\sum_{i=p}^{q} a_i
\qquad\text{for some }1\le p\le q\le k-1,\ \ q-p+1\ge2.
\]
This is OEIS \seqnum{A005243}. In \cite[p.~71]{Er77c}, \cite[p.~83]{ErGr80}, and
\cite[E31]{Gu04}, Hofstadter asked the following question (although Erd\H{o}s notes in
\cite{Er77c} that the question is originally due to Ulam); it also appears as
Problem~\#423 on Bloom's Erd\H{o}s Problems website~\cite{EP}.

\begin{problem}\label{prob:1}
Let $a_1=1$ and $a_2=2$, and for $k\ge 3$ choose $a_k$ to be the least integer
$>a_{k-1}$ that is the sum of at least two consecutive earlier terms of the
sequence. What is the asymptotic behaviour of $(a_n)_{n\ge1}$?
\end{problem}

Empirically, most integers occur. However, a conjecture recorded since 2006 in
the \emph{COMMENTS} section of OEIS \seqnum{A005243} asserts that the sequence has
infinitely many \emph{nonmembers}.

\begin{conjecture}\label{conj:1}
There are infinitely many positive integers that never appear in the sequence
$(a_n)_{n\ge1}$.
\end{conjecture}

The main purpose of this note is to prove Conjecture~\ref{conj:1} by establishing
the following theorem.

\begin{theorem}\label{thm:main}
With $(a_n)_{n\ge1}$ as above and $b_n:=a_n-n$, the sequence $(b_n)_{n\ge1}$ is nondecreasing and
unbounded. Consequently, the set $\{a_n:n\ge1\}$ omits infinitely many positive integers.
\end{theorem}

\begin{remark}\label{rem:asymptotics}
In particular, Theorem~\ref{thm:main} yields a qualitative step toward
Problem~\ref{prob:1}: unboundedness of $b_n$ rules out eventual linearity
$a_n=n+B$ for all sufficiently large $n$. However, it does not by itself
determine the growth rate of $b_n$; for instance, it remains compatible with
$b_n=o(n)$.
\end{remark}


\section{Preliminaries}

The following result is standard; see~\cite[Corollary~1]{SchinzelTijdeman1976}.

\begin{lemma}[\cite{SchinzelTijdeman1976}]\label{lem:schinzel-tijdeman}
If a polynomial $P(x)\in\mathbb Q[x]$ has at
least two simple zeros, then the equation\[
y^m=P(x), \quad x,y \text{ integers},\quad |y|>1,
\]has only finitely many integer solutions $m,x,y$ with $m>2$, $|y|>1$ and these solutions can be found effectively.
\end{lemma}

As a direct consequence , we obtain the following.
\begin{proposition}\label{prop:finite-dioph}
Fix $E\in\mathbb Z$. Then the equation
\[
v^2+v+E=2^m
\]
has only finitely many integer solutions $(v,m)\in\mathbb Z\times\mathbb Z_{>2}$.
\end{proposition}
\begin{proof}
Apply Lemma~\ref{lem:schinzel-tijdeman} to $P(v)=v^2+v+E$ and $y=2$.
The discriminant of $P$ is $1-4E\neq0$ for $E\in\mathbb Z$, hence $P$ has two simple zeros, so the lemma applies.
\end{proof}




\begin{lemma}\label{lem:consecutive-iff-not-2power}
A positive integer $N\geq 3$ is a sum of at least two consecutive positive integers
if and only if $N$ is not a power of $2$.
\end{lemma}

\begin{proof}
If $N=x+(x+1)+\cdots+(x+\ell-1)$ with $\ell\ge2$, then
\[
2N=\ell(2x+\ell-1).
\]
If $\ell$ is odd, then $\ell\mid 2N$ implies $\ell\mid N$, and $\ell>1$ is an odd
divisor of $N$, so $N$ is not a power of $2$.
If $\ell$ is even, then $2x+\ell-1$ is odd and $>1$ and divides $2N$, hence divides
$N$, again forcing $N$ to have an odd divisor $>1$, so $N$ is not a power of $2$.

Conversely, if $N$ is not a power of $2$, then $N$ has an odd divisor $d>1$.
Write $N=dm$. If $m\ge (d+1)/2$, then
\[
N=\Bigl(m-\frac{d-1}{2}\Bigr)+\Bigl(m-\frac{d-1}{2}+1\Bigr)+\cdots+
\Bigl(m+\frac{d-1}{2}\Bigr)
\]
is a sum of $d\ge3$ consecutive positive integers. If $m<(d+1)/2$, then $2m\ge2$ and
\[
N=\Bigl(\frac{d+1}{2}-m\Bigr)+\Bigl(\frac{d+1}{2}-m+1\Bigr)+\cdots+\Bigl(\frac{d+1}{2}+m-1\Bigr)
\]
is a sum of $2m$ consecutive positive integers.
\end{proof}

\section{Proof of Theorem~\ref{thm:main}}




First, we show that $b_n$ is nondecreasing.

\begin{lemma}\label{lem:bn-monotone}
The sequence $(b_n)$ is nondecreasing. In particular, since $b_1=0$, we have
$b_n\ge0$ for all $n$. If $(b_n)$ is bounded above, then there exist integers
$n_0\ge1$ and $B\ge0$ such that
\[
a_n=n+B\qquad\forall\,n\ge n_0.
\]
\end{lemma}

\begin{proof}
Because $(a_n)$ is strictly increasing in integers, $a_{n+1}\ge a_n+1$ for all $n$.
Hence
\[
b_{n+1}-b_n=(a_{n+1}-a_n)-1\ge0,
\]
so $(b_n)$ is nondecreasing. Since $b_1=a_1-1=0$, monotonicity gives $b_n\ge0$.

If $(b_n)$ is bounded above, then a nondecreasing integer sequence must be
eventually constant: there exist $n_0$ and an integer $B$ such that $b_n=B$
for all $n\ge n_0$, i.e.\ $a_n=n+B$ for all $n\ge n_0$. Finally $B\ge0$
because $B=b_n\ge0$ for all $n$. 
\end{proof}

Assume for contradiction that $(b_n)$ is bounded above, and fix $n_0,B$ as in
Lemma~\ref{lem:bn-monotone}. Set
\[
T:=n_0+B.
\]Define the finite  set
\[
\mathcal C:=\Bigl\{\sum_{k=p}^{n_0-1} a_k:\ 1\le p\le n_0\Bigr\},
\]
where the case $p=n_0$ gives the empty sum $0\in\mathcal C$.

\begin{lemma}\label{lem:rep-types}
Assume $a_n=n+B$ for all $n\ge n_0$.
Let $t=a_n$ with $n\ge n_0$ and suppose $t>S_{n_0-1}$ (where $S_m=\sum_{i=1}^m a_i$).
Then any representation
\[
t=\sum_{k=p}^{q} a_k
\qquad(1\le p\le q\le n-1,\ \ q-p+1\ge2)
\]
falls into exactly one of the following two types:
\begin{itemize}
\item[\textup{(A)}] $p\ge n_0$, and then
\[
t=\sum_{m=u}^{v} m
\quad\text{for some integers }u\ge T,\ v\ge u+1.
\]
\item[\textup{(B)}] $p<n_0\le q$, and then
\[
t=C+\sum_{m=T}^{v} m
\quad\text{for some }C\in\mathcal C\text{ and some integer }v\ge T.
\]
\end{itemize}
Moreover, the case $q<n_0$ is impossible whenever $t>S_{n_0-1}$.
\end{lemma}

\begin{proof}
Let $t=\sum_{k=p}^{q} a_k$ with $q-p+1\ge2$ and $q\le n-1$.
If $q<n_0$, then the block lies entirely in $\{1,\dots,n_0-1\}$, hence
\[
t=\sum_{k=p}^{q} a_k\le \sum_{k=1}^{n_0-1} a_k=S_{n_0-1},
\]
contradicting $t>S_{n_0-1}$. Thus $q<n_0$ is impossible, so $q\ge n_0$.

If $p\ge n_0$, then using $a_k=k+B$ for $k\ge n_0$ we obtain
\[
t=\sum_{k=p}^{q} (k+B)=\sum_{k=p}^{q}k+(q-p+1)B.
\]
Set $u:=p+B$ and $v:=q+B$. Then $u\ge n_0+B=T$, and since $q-p+1\ge2$ we have
$v-u+1=q-p+1\ge2$, i.e.\ $v\ge u+1$. Also
\[
\sum_{k=p}^{q}k+(q-p+1)B=\sum_{m=u}^{v} m,
\]
giving type (A).

If instead $p<n_0$, then $q\ge n_0$ as shown above, and we split the block:
\[
t=\sum_{k=p}^{n_0-1} a_k \;+\;\sum_{k=n_0}^{q} a_k.
\]
The first part equals some $C\in\mathcal C$ by definition. For the second part,
\[
\sum_{k=n_0}^{q} a_k=\sum_{k=n_0}^{q}(k+B)=\sum_{m=T}^{q+B} m.
\]
Setting $v:=q+B$ (so $v\ge T$) yields type (B).
These two cases are disjoint as conditions on the indices (either $p\ge n_0$
or $p<n_0$), although the resulting sets of attainable values may overlap.\end{proof}


\begin{lemma}\label{lem:powers-two-force-dioph}
Assume $a_n=n+B$ for all $n\ge n_0$ and write $T=n_0+B$.
Then for every integer $r$ with
\[
2^r\ge \max\{T,\ S_{n_0-1}+1, \ B+3 \},
\]
the number $2^r$ is a term of the sequence and admits a representation of type \textup{(B)}.
Consequently, for each such $r$ there exist $C\in\mathcal C$ and an integer $v\ge T$ such that
\[
2^{r+1}=v^2+v+E_C,\qquad E_C:=2C-(T-1)T.
\]
Moreover, there exists a fixed integer $E$ for which the equation
\[
v^2+v+E=2^m
\]
has infinitely many integer solutions $(v,m)$.
\end{lemma}

\begin{proof}
Fix $r$ with $2^r\ge T=n_0+B$. Since the tail is consecutive integers
$a_n=n+B$ for $n\ge n_0$, taking $n:=2^r-B$ gives $n\ge n_0$ and
\[
a_n=n+B=2^r,
\]
so $2^r$ is indeed a term of the sequence.

Because $n\ge3$ for all sufficiently large $r$, the defining rule of the sequence
provides a representation
\[
2^r=a_n=\sum_{k=p}^{q}a_k
\qquad(1\le p\le q\le n-1,\ \ q-p+1\ge2).
\]
Also $2^r>S_{n_0-1}$ by the hypothesis on $r$, so Lemma~\ref{lem:rep-types} applies.
If the representation were of type (A), then Lemma~\ref{lem:rep-types}(A) would give
\[
2^r=\sum_{m=u}^{v}m
\quad\text{with }v\ge u+1,
\]
a sum of at least two consecutive positive integers, contradicting
Lemma~\ref{lem:consecutive-iff-not-2power} since $2^r$ is a power of $2$.
Thus the representation must be of type (B), so
\[
2^r=C+\sum_{m=T}^{v}m
\quad\text{for some }C\in\mathcal C,\ v\ge T.
\]
Using $\sum_{m=T}^{v}m=\frac{v(v+1)}2-\frac{(T-1)T}2$ and multiplying by $2$ yields
\[
2^{r+1}=v^2+v+\bigl(2C-(T-1)T\bigr)=v^2+v+E_C,
\]
as claimed.

For the final statement, consider the set of pairs
\[
A:=\Bigl\{(r,C)\in\mathbb Z_{>1}\times\mathcal C:\ 
\exists\,v\ge T\ \text{with}\ 2^{r+1}=v^2+v+E_C\Bigr\}.
\]
We have shown: for every sufficiently large $r$,
there exists at least one $C\in\mathcal C$ with $(r,C)\in A$.
Therefore $A$ is infinite. Since $\mathcal C$ is finite, the infinite
pigeonhole principle implies that there exists a fixed $C_\ast\in\mathcal C$
such that $(r,C_\ast)\in A$ for infinitely many $r$.
For each such $r$, choose one corresponding $v$; then setting $m:=r+1$ and
$E:=E_{C_\ast}$ produces infinitely many solutions $(v,m)\in\mathbb Z_{>0}\times\mathbb Z_{>2}$ to
$v^2+v+E=2^m$.
\end{proof}




Now we conclude the main contradiction. By Lemma~\ref{lem:powers-two-force-dioph},
boundedness of $(b_n)$ implies the existence of a fixed integer $E$ for which
$v^2+v+E=2^m$ has infinitely many integer solutions $(v,m)$. This contradicts Proposition~\ref{prop:finite-dioph}. Therefore $(b_n)$ is not bounded above, i.e.\ $(b_n)$
is unbounded. 

Finally, if $\{a_n\}$ omitted only finitely many positive integers, then there would exist
$N_1$ such that every integer $t\ge N_1$ equals some $a_n$. Since $(a_n)$ is strictly increasing,
this would force $a_n=n+B$ for all sufficiently large $n$, i.e.\ boundedness of $b_n$,
contradicting unboundedness. Hence $\{a_n\}$ omits infinitely many integers. This proves Theorem~\ref{thm:main}.\qed

\begin{thebibliography}{99}

\bibitem{EP}
T. F. Bloom, Erd\H{o}s Problem \#423, \url{https://www.erdosproblems.com/423}, accessed 2026-01-15.


\bibitem{Er77c}
Erd\H{o}s, Paul, Problems and results on combinatorial number theory. III. Number theory day (Proc. Conf., Rockefeller Univ.,
New York, 1976) (1977), 43--72.

\bibitem{ErGr80}
Erd\H{o}s, P. and Graham, R., Old and new problems and results in combinatorial number theory. Monographies de L'Enseignement Mathematique (1980).

\bibitem{Gu04}
Guy, Richard K., Unsolved problems in number theory. (2004)

\bibitem{SchinzelTijdeman1976}
A.~Schinzel and R.~Tijdeman,
\newblock On the equation $y^m=P(x)$,
\newblock \emph{Acta Arith.} \textbf{31} (1976), 199--204.


\end{thebibliography}

\end{document}
